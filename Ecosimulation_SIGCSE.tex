% This is "sig-alternate.tex" V1.9 April 2009
% This file should be compiled with V2.4 of "sig-alternate.cls" April 2009
%
% This example file demonstrates the use of the 'sig-alternate.cls'
% V2.4 LaTeX2e document class file. It is for those submitting
% articles to ACM Conference Proceedings WHO DO NOT WISH TO
% STRICTLY ADHERE TO THE SIGS (PUBS-BOARD-ENDORSED) STYLE.
% The 'sig-alternate.cls' file will produce a similar-looking,
% albeit, 'tighter' paper resulting in, invariably, fewer pages.
%
% ----------------------------------------------------------------------------------------------------------------
% This .tex file (and associated .cls V2.4) produces:
%       1) The Permission Statement
%       2) The Conference (location) Info information
%       3) The Copyright Line with ACM data
%       4) NO page numbers
%
% as against the acm_proc_article-sp.cls file which
% DOES NOT produce 1) thru' 3) above.
%
% Using 'sig-alternate.cls' you have control, however, from within
% the source .tex file, over both the CopyrightYear
% (defaulted to 200X) and the ACM Copyright Data
% (defaulted to X-XXXXX-XX-X/XX/XX).
% e.g.
% \CopyrightYear{2007} will cause 2007 to appear in the copyright line.
% \crdata{0-12345-67-8/90/12} will cause 0-12345-67-8/90/12 to appear in the copyright line.
%
% ---------------------------------------------------------------------------------------------------------------
% This .tex source is an example which *does* use
% the .bib file (from which the .bbl file % is produced).
% REMEMBER HOWEVER: After having produced the .bbl file,
% and prior to final submission, you *NEED* to 'insert'
% your .bbl file into your source .tex file so as to provide
% ONE 'self-contained' source file.
%
% ================= IF YOU HAVE QUESTIONS =======================
% Questions regarding the SIGS styles, SIGS policies and
% procedures, Conferences etc. should be sent to
% Adrienne Griscti (griscti@acm.org)
%
% Technical questions _only_ to
% Gerald Murray (murray@hq.acm.org)
% ===============================================================
%
% For tracking purposes - this is V1.9 - April 2009

\documentclass{sig-alternate}
\usepackage{color}
\usepackage{listings}
\usepackage{url}
\usepackage{algorithm}
\usepackage{algorithmic}
\usepackage{multirow}
\usepackage{booktabs}
\newcommand{\FIXME}[1]{{\color{red}\{FIXME #1\}}}
\newcommand{\INDSTATE}[1][1]{\STATE\hspace{#1\algorithmicindent}}
\newcommand{\EcoSim}{\texttt{EcoSim~}}
\urldef\ecosimPath\path{http://ecosimulation.com}

\begin{document}
%
% --- Author Metadata here ---
\conferenceinfo{SIGCSE}{2012 Raleigh, NC, USA}
%\CopyrightYear{2007} % Allows default copyright year (20XX) to be over-ridden - IF NEED BE.
%\crdata{0-12345-67-8/90/01}  % Allows default copyright data (0-89791-88-6/97/05) to be over-ridden - IF NEED BE.
% --- End of Author Metadata ---

\title{Parallel Programming in Elementary School}
%
% You need the command \numberofauthors to handle the 'placement
% and alignment' of the authors beneath the title.
%
% For aesthetic reasons, we recommend 'three authors at a time'
% i.e. three 'name/affiliation blocks' be placed beneath the title.
%
% NOTE: You are NOT restricted in how many 'rows' of
% "name/affiliations" may appear. We just ask that you restrict
% the number of 'columns' to three.
%
% Because of the available 'opening page real-estate'
% we ask you to refrain from putting more than six authors
% (two rows with three columns) beneath the article title.
% More than six makes the first-page appear very cluttered indeed.
%
% Use the \alignauthor commands to handle the names
% and affiliations for an 'aesthetic maximum' of six authors.
% Add names, affiliations, addresses for
% the seventh etc. author(s) as the argument for the
% \additionalauthors command.
% These 'additional authors' will be output/set for you
% without further effort on your part as the last section in
% the body of your article BEFORE References or any Appendices.

\numberofauthors{4} %  in this sample file, there are a *total*
% of EIGHT authors. SIX appear on the 'first-page' (for formatting
% reasons) and the remaining two appear in the \additionalauthors section.
%
\author{
% You can go ahead and credit any number of authors here,
% e.g. one 'row of three' or two rows (consisting of one row of three
% and a second row of one, two or three).
%
% The command \alignauthor (no curly braces needed) should
% precede each author name, affiliation/snail-mail address and
% e-mail address. Additionally, tag each line of
% affiliation/address with \affaddr, and tag the
% e-mail address with \email.
%
% 1st. author
\alignauthor
Chris Gregg\\
       \affaddr{University of Virginia}\\
       \affaddr{Department of Computer Science}\\
       \affaddr{Charlottesville, VA}\\
% 2nd. author
\alignauthor
Luther Tychonievich\\
       \affaddr{University of Virginia}\\
       \affaddr{Department of Computer Science}\\
       \affaddr{Charlottesville, VA}\\
% 3rd. author
\alignauthor Kim Hazelwood\\
       \affaddr{University of Virginia}\\
       \affaddr{Department of Computer Science}\\
       \affaddr{Charlottesville, VA}\\
\and  % use '\and' if you need 'another row' of author names
% 4th. author
\alignauthor James Cohoon\\
       \affaddr{University of Virginia}\\
       \affaddr{Department of Computer Science}\\
       \affaddr{Charlottesville, VA}\\
}
% There's nothing stopping you putting the seventh, eighth, etc.
% author on the opening page (as the 'third row') but we ask,
% for aesthetic reasons that you place these 'additional authors'
% in the \additional authors block, viz.
%\date{30 July 1999}
% Just remember to make sure that the TOTAL number of authors
% is the number that will appear on the first page PLUS the
% number that will appear in the \additionalauthors section.

\maketitle
\begin{abstract}
Traditional introductory programming classes focus on teaching sequential programming
skills using conventional programming languages and single-threaded applications.
It isn't generally until much later in a student's programming education that he or she learns
about parallel programming and associated topics such as race conditions, locks, or data
consistency.  With the increased popularity of 
multicore CPUs and GPUs capable of GPGPU computing, there is a greater 
need for programmers who are not only proficient in parallel programming, but who are
not burdened by an inclination towards trying to solve a problem in a sequential fashion, with
parallelism tacked on as an afterthought.

Pedagogically, there is a case to be made that teaching parallelism first is an important 
step towards educating tomorrow's programmers for the challenges of programming multicore and GPGPU
systems.  We present an overview of a five-day introductory parallel programming course we 
taught to a group of nine and ten year-olds, using a near-natural language syntax
parallel programming language we created, targeted towards students with no previous programming
experience.  Our language is simple but powerful and consists 
of a simulated parallel programming environment and the ability to run or step through programs.

We provide examples of student-written code that demonstrates their understanding of some basic
parallel programming concepts, and we describe the overall course goal and specific lesson plans
geared towards teaching students how to ``think parallel.''
\end{abstract}

% A category with the (minimum) three required fields
\category{H.4}{Information Systems Applications}{Miscellaneous}
%A category including the fourth, optional field follows...
\category{D.2.8}{Software Engineering}{Metrics}[complexity measures, performance measures]

\terms{Languages, Design}

\keywords{Concurrent, distributed, and parallel languages, Instructional Design,
Introductory Programming, Pedagogy, Education}

\section{Introduction}
Introductory programming classes are almost universally taught 
using languages designed primarily for single-threaded applications.  Multi-threaded or
parallel programming concepts are considered advanced, and it is rare that students learn about
parallel programming before a second or third programming course.  Indeed, most colleges and
universities in the United States provide a single parallel programming course available 
to upper level undergraduates or graduate students\FIXME{citestats}, and such courses are
almost always optional in the computer science curriculum.  In many cases, only students who
are interested in high performance computing are ever exposed to parallel programming, and the
average programmer never receives any traditional instruction in parallel programming at all.
Additionally, when students do learn parallel programming, many have difficulties transitioning
from a sequential-programming mentality to a parallel programming mentality, especially as parallel
programming is considered ``hard'' by many students and instructors 
alike.~\cite{parallelExpectations}

Within the last five years, multicore computing has become the \emph{de facto} standard on
desktops and laptops, and General Purpose GPU (GPGPU) computing has matured such that multi-core 
GPUs can be programmed with minimal extensions to traditional languages such as C++ and 
Python~\cite{gpgpuLanguages}.  The trend towards increasing cores to program on a
single machine does not show any signs of abating in the near future~\cite{multicoreTrends}, and
therefore parallel programming skills are going to become increasingly important.  Programmers
must not only thoroughly understand parallel programming concepts such as race conditions,
atomicity, synchronization, and deadlock, but they must be able to look at a computing problem
and think of solutions that utilize parallel processes.

With the disconnect between sequential-only introductory programming classes and the necessity for
programming students to learn parallel programming concepts and methods in mind, we developed
an introductory parallel programming course that specifically targeted novice programmers.  We
designed a language, called \emph{EcoSim}\footnote{\emph{EcoSim} is so-named because the original 
class we taught with it focused on an \emph{eco}logical \emph{sim}ulation.}\FIXME{Are we going to 
name the language officially?}, that simulates a parallel programming environment and has a highly
accessible natural language syntax.  Programs written in \emph{EcoSim} have the ability to exhibit race 
conditions, allow both atomic and non-atomic variable assignment, and show increased performance
when the number of cores is increased.  It is a turing-complete language, and
contains a number of basic functions geared towards making the programs interesting for novice
programmers.  Figure \ref{fig:exampleProgram} shows an example \emph{EcoSim} program that defines and
draws ten green ``plants'' on the screen, where the plants are represented by circles of radius 10.
Figure \ref{fig:ecosimScreencap} shows the \emph{EcoSim} development environment, which includes
a code window, settings, a console window with output messages, and a window for graphical
objects.
\begin{figure}
\begin{algorithmic}[1]
%\INDSTATE[x]{code} indents x number of tabs, \INDSTATE{code} indents a single tab
\STATE{a plant has}
  \INDSTATE{a position}
  \INDSTATE{size, a number}
  \INDSTATE{a color}
\STATE{}
\STATE{create 10 plant and for each}
  \INDSTATE{do in order}
  \INDSTATE{replace the plant's color with green}
  \INDSTATE{replace the plant's size with 10}
\end{algorithmic} 
\caption{A simple \emph{EcoSim} program to define and create ten green ``plants'' on the screen.}
\label{fig:exampleProgram} 
\end{figure}

\begin{figure}
\centerline{\includegraphics[width=.49\textwidth]{figures/EcosimScreencap.png}}
\caption{The \emph{EcoSim} web-based integrated development environment hosted at
\ecosimPath{}.  Code is written and debugged in the top left window, settings are on
the top right, a console with runtime and debug information is below the settings, and the
main window shows the graphical output of the program.}
\label{fig:ecosimScreencap}
\end{figure}

We had three overarching goals in mind for the course we designed around \emph{EcoSim}:
\begin{enumerate}
\item Introduce the students to simple parallel programming ideas using multiple processors.
\item Provide interesting parallel programming examples the students could easily modify and
learn from.
\item Teach the students to ``think parallel'' about computing problems we gave them, or that
they thought up on their own.
\FIXME{should we include the define the task/describe a solution/tell the computer here?}
\end{enumerate}

We presented the course, titled, ``Programming the Computers of the Future'' to two classes of 
eighteen 4th and 5th grade (9 and 10 year old) students during a five-day enrichment program.
Each class period was two hours long, and the students had a week between classes, although they
could access the programming development environment online to continue learning independently.
None of the students had significant prior programming experience.
We based the course curriculum on creating a simulated ecosystem, starting with simple objects
such as stationary plants that could grow in place, and eventually creating herbivores and
carnivores that could move about the screen.
Our lessons included group exercises that introduced parallel programming concepts and general
programming-style problem solving, and each lesson included example \emph{EcoSim} programs with
time for the students to modify or attempt to create new programs on their own.

We had many successes in our pilot course:

\begin{enumerate}
\item Exit surveys collected from the students in both classes showed enthusiastic responses to the
class, and students reported that they learned a number of programming concepts.
\item Student code examples show that by the end of the class students were familiar with the
language and were able to write programs that took advantage of parallel concepts.
\item After one or two classes the students
felt comfortable with basic concepts of \emph{EcoSim} and were able to write rudimentary
parallel programs without trouble.  By the end of the course, a number of students designed
and implemented creative programs that highlighted the parallel nature of the language.
\end{enumerate}

\section{Background and Related Work}

Parallel computing has a long history, dating back to 1955 and the IBM 704
and its ability to compute parallel arithmetic~\cite{hockney1988parallel}.  Amdahl's law,
defining the maximum possible speedup due to parallelization, was coined in 
1967~\cite{amdahl1967validity}, and multiprocessor mainframes and multinode distributed
computing platforms provided most of the world's parallel processing until the early 2000s.
However, even with multiprocessor systems, programmers were first taught how to write
sequential applications, generally learning parallel programming concepts for specific
computers or platforms.  The microcomputer explosion of the late 70s and early 80s
ensured that most programmers were exposed to uniprocessor machines as their first computers,
and thus their first programming experiences were with sequential programming languages as well.
Today, multicore desktop and laptop computers are ubiquitous, and in order to make the most
efficient use of these computers, parallel programming is necessary.  Furthermore, when novice
programmers sit down to write their first code, it is using a a parallel computer.

There are numerous programming languages available for desktop parallel programming.  Many of
these languages are extensions, libraries, or APIs built on top of sequential languages such 
as C or Fortran (e.g., OpenMP, CUDA, 
OpenCL, Intel Thread Building Blocks, pthreads, Cilk,
Co-array Fortran, and Unified Parallel C),
requiring a novice programmer to first become proficient in a sequential language before
tackling the parallel programming concepts.  While this does not necessarily hinder a student's
overall programming ability, parallel programming tends to receive less importance than simply
learning the sequential aspects of the language.  There have been a number of studies on teaching 
parallel programming concepts using traditional languages at the 
undergraduate level~\cite{freshmanParallel,undergraduateParallel,gridPortal} 
and at least one at the secondary school level~\cite{highSchoolParallel}.

There are also languages designed for parallel programming, but they tend have advanced
syntax and be targeted towards
students already proficient at programming in general (e.g., X10\cite{X10}, 
NESL~\cite{nesl-impl-94}, and Go~\cite{GoLanguage}).  It would be hard to suggest any of these
languages to an absolute beginner programmer.


\section{EcoSim: An Introductory Parallel Programming Language}
\begin{enumerate}
\item Overview
\item Language semantics
\item Language syntax (natural language)
\item The graphical interface (including collision detection, wall detection, etc.)
\item Things that set it apart from other parallel programming languages
\item Examples
\end{enumerate}

\section{Course Overview and Lesson Plans}
The pilot course we created was for fourth and fifth grade students in an enrichment program
that is run through our university.  We designed the course and \emph{EcoSim} concurrently,
and both were targeted for our audience of self-selected primary school students with no
prior formal programming experience.
\begin{enumerate}
\item Overview (ecosystem in parallel)
\item Starting to "think parallel" -- student sort
\item Ingraining the idea of multiple processors working independently to solve the same problem.
\item Constantly let kids show off what they have accomplished
\item the idea of "when bored", for some, for all
\item atomicity and race conditions (class example on board -- roll/read/roll/write)
\item St. Matthew Island
\end{enumerate}

\subsection{Ecosystem in Parallel}
The original conception of the pilot course was, simply, ``Let's teach fourth and fifth graders
about parallel programming.''  We decided on an ``ecosystem'' theme for the course, based on a
number of reasons.  First, students at this level are familiar with real-life ecosystems, and we
felt that they would find the topic interesting.  Second, ecosystems have a number of 
embarrassingly parallel characteristics; for example, in a forest there are multiple copies
of trees which can each be handled independently and in parallel.  
Finally, we knew we could model a simple ecosystem 
and then build upon the original model to make it more complex.  Starting with a forest of
stationary plants that have a single ``grow'' characteristic, we added motile herbivores that
consumed the plants.  We then added the ability for the inhabitants to reproduce and gave them
the ability to die from starvation, and then eventually we added carnivores as well.  
By the end of the course students had expanded the ecosystem to include plants that only
grew during the day, hunters, and even carnivorous and poisonous plants.  

Students quickly learned the importance of initial conditions and parameters, both from a 
computational perspective and a scientific one.  For example, students found that starting 
ten thousand herbivores in a field with only ten plants
not only slows the computer to a crawl, but the herbivores quickly decimate the plant population
and start to die from starvation.  We spent a number of classes discussing and
modeling the intriguing real-life case of a herd of reindeer who overpopulated a remote
island in Alaska and subsequently died out~\cite{klein1968introduction,stMatthewIsland}, 
and with the \emph{EcoSim} model the students could adjust the parameters to find an
equilibrium that would have allowed the reindeer to survive.

\subsection{Getting the students to ``think parallel''}
At the beginning of each class period and before writing any code, we first introduced the 
students to a parallel programming concept in a full-class discussion, usually with an activity.
For instance, on the first day of class we introduced the students to the difference in
computational time between parallel and sequential processes by having them sort themselves
by height.  First, we allowed the students to line themselves up by height, all at once (the
parallel method), and we timed this; it took roughly forty-five seconds for a class of eighteen.
Next, we re-randomized the class and assigned one student to be the ``processor,'' in charge
of sorting the students two at a time.  Unsurprisingly, this took over three minutes, and this
led to a fruitful discussion on why parallel processing can be faster.

Table \ref{tab:group-activities} shows the group activities we conducted an their associated
parallel processing concept or concepts.  During and after each activity, we discussed the
associated concept and in most cases we then wrote a simple program in \emph{EcoSim} that
demonstrated the idea.
For example, after the race condition activity, we wrote the programs in Figure 
\ref{fig:atomic-example}, which demonstrate a race condition stemming from allowing multiple
processors to complete the \texttt{color} statements in any order.

\begin{table*}
\centering \begin{tabular}{p{11cm} | p{5cm}} 
\toprule
Group Activity               &  Parallel Programming Concept \\ \midrule
Students sort themselves, and then one student sorts everyone.     & Parallel speedup \\ \hline
Everyone shares a pen to write on the whiteboard to increment a number. & Locks / Atomicity  \\ \hline
Students roll a set of dice until they roll a specific combination.  Then they
look at the board for a number, increment, and call out the new number, which is
written on the board.  & Race Conditions \\ \hline
All students start with a number, and half hand to their neighbor to add together.
This continues until one student has the total sum. & Reduction and Divide/Conquer  \\
\bottomrule 
\end{tabular}
\caption{Group activities.}
\label{tab:group-activities}
\end{table*} 
     
\begin{figure}
\begin{algorithmic}[1]
%\INDSTATE[x]{code} indents x number of tabs, \INDSTATE{code} indents a single tab
\item[{\bf In order:}]
\STATE{a moth has}
  \INDSTATE{a position}
  \INDSTATE{a color}
\STATE{}
\STATE{create 10 moth and for each}
  \INDSTATE{do in order}
  \INDSTATE[2]{replace the moth's color with gray}
  \INDSTATE[2]{replace the moth's color with black}

\item[{\bf In any order:}]
\STATE{a moth has}
  \INDSTATE{a position}
  \INDSTATE{a color}
\STATE{}
\STATE{create 10 moth and for each}
  \INDSTATE{do in any order}
  \INDSTATE[2]{replace the moth's color with gray}
  \INDSTATE[2]{replace the moth's color with black}
\end{algorithmic} 
\caption{Example \emph{EcoSim} programs that demonstrate race conditions.  In the
in order program, all moths end up black, while in the out of order program
the final color is dependent on a race condition.}
\label{fig:raceConditions} 
\end{figure}

\section{Student Work and Outcomes}

\section{Conclusions}
Conclusions

%ACKNOWLEDGMENTS are optional
%\section{Acknowledgments}
%This section is optional; it is a location for you
%to acknowledge grants, funding, editing assistance and
%what have you.  In the present case, for example, the
%authors would like to thank Gerald Murray of ACM for
%his help in codifying this \textit{Author's Guide}
%and the \textbf{.cls} and \textbf{.tex} files that it describes.

%
% The following two commands are all you need in the
% initial runs of your .tex file to
% produce the bibliography for the citations in your paper.
\bibliographystyle{abbrv}
\bibliography{ecosim}  % sigproc.bib is the name of the Bibliography in this case
% You must have a proper ".bib" file
%  and remember to run:
% latex bibtex latex latex
% to resolve all references
%
% ACM needs 'a single self-contained file'!
%

\end{document}
